% mnras_template.tex 
%
% LaTeX template for creating an MNRAS paper
%
% v3.0 released 14 May 2015
% (version numbers match those of mnras.cls)
%
% Copyright (C) Royal Astronomical Society 2015
% Authors:
% Keith T. Smith (Royal Astronomical Society)

% Change log
%
% v3.0 May 2015
%    Renamed to match the new package name
%    Version number matches mnras.cls
%    A few minor tweaks to wording
% v1.0 September 2013
%    Beta testing only - never publicly released
%    First version: a simple (ish) template for creating an MNRAS paper

%%%%%%%%%%%%%%%%%%%%%%%%%%%%%%%%%%%%%%%%%%%%%%%%%%
% Basic setup. Most papers should leave these options alone.
\documentclass[fleqn,usenatbib]{mnras}

% MNRAS is set in Times font. If you don't have this installed (most LaTeX
% installations will be fine) or prefer the old Computer Modern fonts, comment
% out the following line
\usepackage{newtxtext,newtxmath}
% Depending on your LaTeX fonts installation, you might get better results with one of these:
%\usepackage{mathptmx}
%\usepackage{txfonts}

% Use vector fonts, so it zooms properly in on-screen viewing software
% Don't change these lines unless you know what you are doing
\usepackage[T1]{fontenc}

% Allow "Thomas van Noord" and "Simon de Laguarde" and alike to be sorted by "N" and "L" etc. in the bibliography.
% Write the name in the bibliography as "\VAN{Noord}{Van}{van} Noord, Thomas"
\DeclareRobustCommand{\VAN}[3]{#2}
\let\VANthebibliography\thebibliography
\def\thebibliography{\DeclareRobustCommand{\VAN}[3]{##3}\VANthebibliography}


%%%%% AUTHORS - PLACE YOUR OWN PACKAGES HERE %%%%%

% Only include extra packages if you really need them. Common packages are:
\usepackage{graphicx}	% Including figure files
\usepackage{amsmath}	% Advanced maths commands
\usepackage{amssymb}	% Extra maths symbols

%%%%%%%%%%%%%%%%%%%%%%%%%%%%%%%%%%%%%%%%%%%%%%%%%%

%%%%% AUTHORS - PLACE YOUR OWN COMMANDS HERE %%%%%

% Please keep new commands to a minimum, and use \newcommand not \def to avoid
% overwriting existing commands. Example:
%\newcommand{\pcm}{\,cm$^{-2}$}	% per cm-squared

%%%%%%%%%%%%%%%%%%%%%%%%%%%%%%%%%%%%%%%%%%%%%%%%%%

%%%%%%%%%%%%%%%%%%% TITLE PAGE %%%%%%%%%%%%%%%%%%%

% Title of the paper, and the short title which is used in the headers.
% Keep the title short and informative.
\title[Importance sampling with normalizing flows]{Rapidly combining cosmological probes with importance sampling and normalizing flows }

% The list of authors, and the short list which is used in the headers.
% If you need two or more lines of authors, add an extra line using \newauthor
\author[B. Thorne]{
B. Thorne,$^{1}$\thanks{E-mail: blthorne@ucdavis.edu}
L. Knox,$^{1}$
\\
% List of institutions
$^{1}$Department of Physics, University of California, One Shields Avenue, Davis, CA 95616, USA\\
}

% These dates will be filled out by the publisher
\date{Accepted XXX. Received YYY; in original form ZZZ}

% Enter the current year, for the copyright statements etc.
\pubyear{2021}

% Don't change these lines
\begin{document}
\label{firstpage}
\pagerange{\pageref{firstpage}--\pageref{lastpage}}
\maketitle

% Abstract of the paper
\begin{abstract}
This is an abstract
\end{abstract}

% Select between one and six entries from the list of approved keywords.
% Don't make up new ones.
\begin{keywords}
keyword1 -- keyword2 -- keyword3
\end{keywords}

%%%%%%%%%%%%%%%%%%%%%%%%%%%%%%%%%%%%%%%%%%%%%%%%%%

%%%%%%%%%%%%%%%%% BODY OF PAPER %%%%%%%%%%%%%%%%%%

\section{Introduction}

\begin{itemize}
    \item Difficulty of importance sampling in datasets with poor coverage of true posterior
    \item General applicability to Bayesian inference: MCMC techniques etc
\end{itemize}

\section{Methods}

\subsection{Normalizing Flows}
\label{sec:normalizing_flows} % used for referring to this section from elsewhere

Normalizing flows have gained much attention in recent years for their ability to model extremely complicated posterior distributions in terms of simple, tractable base distributions. Typically, they have been applied in the machine learning community to model the distributions of complicated datasets such as natural images \cite{ho/etal:2019, grathwohl/etal:2018, kingma/dhariwal:2018}, video \cite{kumar/etal:2019}, and audio \cite{kim/etal:2018}. In the rest of this section we will introduce the formalism of normalizing flows, for additional details we refer readers to reviews of this subject \cite{kobyzev/etal:2019, papamakarios/etal:2019}.

The central idea is to start with a simple distribution $q_z(z)$, take samples from this distribution $z$, and apply a series of $n$ transformations $f_i(z)$ with $i\in n$, such that the $n^{\rm th}$ iterate is given by $x = f_n \circ f_{n-1} \circ \dots \circ f_1(z)$. For this overall transformation to be a \emph{flow} we require the individual transformations to be both differentiable and invertible, a consequence of which is that their composition will also be differentiable and invertible.

One can think of these transformations as stretching and shaping the simple space in order to fit a more complicated distribution. The details of this stretching and shaping are encoded the determinant of the transformations, $J_{f_i}(z)$. The log likelihood of $x$ can be written in terms of the Jacobians of the individual transformations by a change of variables:
\begin{equation}
\label{eq:log_prob}
    \log q_x(x) = \log q_z(f^{-1}(x)) - \sum_{i=1}^n \log | \det J_f(f^{-1}(x)) |.
\end{equation}
For a given flow, $f$, this formalism allows us to perform two tasks:
\begin{itemize}
\item[i)] we can sample $x$ by first sampling $z \sim p(z)$ and pushing this through the transformation: $x = f(z)$, 
\item[ii)] Evaluating the log probability for a given $x$, which requires calculating the inverse transformation $z = f^{-1}(x)$ as well as the log determinant of the Jacobian in Equation \ref{eq:log_prob}, and the log probability of the base distribution $q(z)$.  
\end{itemize}

The choice of a multivariate Gaussian base distribution makes the sampling and evaluation of $q(z)$ trivial. However, the choice of $f$ involves trade-offs, as it is non-trivial to find transformations that permit efficient evaluations of the Jacobian, whilst also having fast forward and reverse directions. The application being pursued will determine which of these properties is prioritized \cite{papamakarios/etal:2019}.

It is a common choice to model $f$ with a neural network, with some weights $\theta$, and we use the notation $q_x(x; \theta)$ to denote the resulting model for the distribution of $x$.

\begin{itemize}
\item[i)] Describe planar and radial flows, NICE architectures, MADE, MAF, and IAF
\item[ii)] Explain parametrization we chose and maximization objective.
\end{itemize}
\subsection{Importance Sampling}
\label{sec:importance_sampling}



\section{Data \& Training}
\label{sec:data}
We will apply this technique to a combination of Planck and BAO data. In order to demonstrate the potential speedups involved, we first fit a normalizing flow to the 

\subsection{CMB Data}


We use a publicly available posterior MCMC chain {\tt Plik\_TTTEEE\_lowl+lowE}, which we download from the Planck Legacy Archive \footnote{\url{https://pla.esac.esa.int/#home}}

\subsection{BAO Data}
\label{sec:bao_data}

Baryon acoustic oscillations (BAOs) provide a \emph{standard ruler} that may be used to measure the expansion history of the Universe. At recombination, acoustic waves travelling through the photon-baryon plasma are frozen into the matter distribution, defining a characteristic scale that is present both in the CMB, and in the low-redshift distribution of galaxies. 

This characteristic scale shows up as a sharp feature in the correlation function of galaxies at the comoving sound horizon scale, $r_s$, or as an oscillatory feature in their power spectrum, well above the virialization scale at which non-linear physics becomes important. The measurement of the BAO scale therefore provides a robust model-independent measurement of the expansion rate and angular diameter distance as a function of redshift. 

BAO measurements are fequently used in conjunction with CMB datasets because they break key parameter degeneracies 

\section{Results}
\label{sec:results}


\section{Conclusions \& Discussion}
\label{sec:conclusions}


\section*{Acknowledgements}

We acknowledge the use of the GPU allocation at NERSC.

%%%%%%%%%%%%%%%%%%%%%%%%%%%%%%%%%%%%%%%%%%%%%%%%%%
\section*{Data Availability}

 
All the data used in this study is available on the Planck Legacy Archive: \url{https://pla.esac.esa.int/#home}.




%%%%%%%%%%%%%%%%%%%% REFERENCES %%%%%%%%%%%%%%%%%%

% The best way to enter references is to use BibTeX:

\bibliographystyle{mnras}
\bibliography{library} % if your bibtex file is called example.bib


% Alternatively you could enter them by hand, like this:
% This method is tedious and prone to error if you have lots of references
%\begin{thebibliography}{99}
%\bibitem[\protect\citeauthoryear{Author}{2012}]{Author2012}
%Author A.~N., 2013, Journal of Improbable Astronomy, 1, 1
%\bibitem[\protect\citeauthoryear{Others}{2013}]{Others2013}
%Others S., 2012, Journal of Interesting Stuff, 17, 198
%\end{thebibliography}

%%%%%%%%%%%%%%%%%%%%%%%%%%%%%%%%%%%%%%%%%%%%%%%%%%

%%%%%%%%%%%%%%%%% APPENDICES %%%%%%%%%%%%%%%%%%%%%

\appendix

\section{Some extra material}

If you want to present additional material which would interrupt the flow of the main paper,
it can be placed in an Appendix which appears after the list of references.

%%%%%%%%%%%%%%%%%%%%%%%%%%%%%%%%%%%%%%%%%%%%%%%%%%


% Don't change these lines
\bsp	% typesetting comment
\label{lastpage}
\end{document}

% End of mnras_template.tex
